\subsection{Q2 --- Paramètres configurables du CPU O3 (DerivO3CPU)}

\begin{formal}
\textbf{Objectif.}\\
Identifier des paramètres configurables du processeur \textit{out-of-order} de gem5 (\texttt{DerivO3CPU}) et préciser, pour chacun, sa valeur par défaut ainsi que son rôle.
\end{formal}

\subsubsection*{Méthode (où chercher les paramètres)}
Sur les machines ENSTA, les paramètres du CPU O3 sont définis dans le fichier Python
\texttt{O3CPU.py} (répertoire \texttt{src/cpu/o3} de gem5).  
Pour lister rapidement les paramètres et leurs valeurs par défaut, on se place dans le dossier du CPU O3 puis on filtre les lignes contenant \texttt{Param.} :

\begin{verbatim}
cd /auto/g/gbusnot/ES201/tools/TP5/gem5-stable/src/cpu/o3
grep -n "Param\." O3CPU.py
\end{verbatim}

Le \texttt{grep -n} affiche les numéros de ligne, ce qui permet de retrouver facilement la définition exacte des paramètres dans \texttt{O3CPU.py}.

\subsubsection*{Sélection de 5 paramètres (valeur par défaut + impact)}
Nous avons choisi des paramètres liés (i) à la fenêtre OoO (ROB / IQ), (ii) au sous-système mémoire spéculatif (Load/Store Queues), et (iii) à la prédiction de branchement, car ce sont des éléments déterminants pour l’IPC et la performance globale.

\begin{table}[H]
\centering
\caption{Paramètres \texttt{DerivO3CPU} (extraits de \texttt{O3CPU.py})}

\scriptsize
\setlength{\tabcolsep}{4pt}
\renewcommand{\arraystretch}{1.35}

% Ajuste clave: Paramètre más estrecha, Valeur par défaut más ancha
\begin{tabular}{@{}p{0.16\linewidth} p{0.26\linewidth} p{0.54\linewidth}@{}}
\toprule
\textbf{Paramètre} & \textbf{Valeur par défaut} & \textbf{Rôle / impact (résumé)} \\
\midrule

\texttt{numROBEntries} & \texttt{192} &
Taille du \textit{Reorder Buffer} : nb. d’instructions ``en vol''.
Plus grand $\Rightarrow$ meilleure exploitation de l’ILP et masquage de latences. \\
\addlinespace[0.45em]

\texttt{numIQEntries} & \texttt{64} &
Taille de l’\textit{Issue Queue} : instructions prêtes à être émises.
Limite la fenêtre effective (même si le ROB est grand). \\
\addlinespace[0.45em]

\texttt{LQEntries} & \texttt{32} &
\textit{Load Queue} : nb. de loads suivis/pendants en OoO.
Important pour le recouvrement mémoire et la gestion des dépendances. \\
\addlinespace[0.45em]

\texttt{SQEntries} & \texttt{32} &
\textit{Store Queue} : nb. de stores en vol avant écriture en cache/mémoire.
Impact sur le débit mémoire et les dépendances load-after-store. \\
\addlinespace[0.45em]

\texttt{branchPred} & \url{TournamentBP(numThreads=Parent.numThreads)} &
Prédicteur de branchements par défaut.
Une mauvaise prédiction provoque des \textit{flush/squash} et dégrade l’IPC. \\

\bottomrule
\end{tabular}

\normalsize
\end{table}