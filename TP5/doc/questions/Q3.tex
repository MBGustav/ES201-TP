\subsection{Q3 --- Valeurs par défaut des caches (L1I, L1D, L2)}

\begin{formal}
\textbf{Objectif.}\\
Retrouver et reporter les valeurs par défaut des paramètres de cache (tailles, associativités et taille de ligne) utilisées par \texttt{se.py} lorsque aucune option n’est passée en ligne de commande.
\end{formal}

\subsubsection*{Méthode}
D’après l’énoncé, les valeurs par défaut sont définies dans \texttt{\$GEM5/configs/common/Options.py}.
Dans ce fichier, elles apparaissent dans les lignes \texttt{parser.add\_option(... default=...)}.  
Ces \texttt{default=} correspondent aux paramètres effectivement utilisés par \texttt{se.py} si l’on ne fournit pas d’options explicites au lancement.

\medskip
\noindent Exemples de recherche (optionnel) :
\begin{verbatim}
grep -n "l1d_size"   $GEM5/configs/common/Options.py
grep -n "l1i_size"   $GEM5/configs/common/Options.py
grep -n "l2_size"    $GEM5/configs/common/Options.py
grep -n "cacheline_size" $GEM5/configs/common/Options.py
\end{verbatim}

\subsubsection*{Valeurs par défaut (Options.py)}
\scriptsize
\setlength{\tabcolsep}{4pt}
\renewcommand{\arraystretch}{1.25}

\begin{table}[H]
\centering
\caption{Paramètres de cache par défaut (extraits de \texttt{Options.py})}
\begin{tabular}{@{}p{0.16\linewidth} p{0.44\linewidth} p{0.34\linewidth}@{}}
\toprule
\textbf{Niveau} & \textbf{Option gem5} & \textbf{Valeur par défaut} \\
\midrule
L1D & \texttt{--l1d\_size} & \texttt{64kB} \\
L1D & \texttt{--l1d\_assoc} & \texttt{2} (2-way) \\
\addlinespace[0.35em]

L1I & \texttt{--l1i\_size} & \texttt{32kB} \\
L1I & \texttt{--l1i\_assoc} & \texttt{2} (2-way) \\
\addlinespace[0.35em]

L2 & \texttt{--l2\_size} & \texttt{2MB} \\
L2 & \texttt{--l2\_assoc} & \texttt{8} (8-way) \\
\addlinespace[0.35em]

Global & \texttt{--cacheline\_size} & \texttt{64B} \\
\bottomrule
\end{tabular}
\end{table}
\normalsize

\subsubsection*{Remarque}
La taille de ligne est donnée par un paramètre global (\texttt{cacheline\_size}).  
Par défaut, elle s’applique donc à la L1D, la L1I et la L2 (sauf si une configuration spécifique la surcharge ailleurs).