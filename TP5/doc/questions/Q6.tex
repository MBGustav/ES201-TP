\subsection{Q6 --- Speedup par rapport à la configuration à 1 thread}

\begin{formal}
\textbf{Énoncé (Q6).}\\
Déduire le speedup par rapport à la configuration à 1 thread.
\end{formal}

\subsubsection*{Définition}
À fréquence constante, le temps d’exécution est proportionnel au nombre de cycles.  
On en déduit le speedup par rapport à 1 thread :
\[
\text{Speedup}(T) \;=\; \frac{cycles_{app}(1)}{cycles_{app}(T)},
\]
où $cycles_{app}(T)$ est défini en Q5 comme le maximum de \texttt{numCycles} (cœur critique). Les valeurs de $cycles_{app}(T)$ sont celles déjà reportées en Q4 (Tables~\ref{tab:q4_cycles_l2} et~\ref{tab:q4_cycles_nol2}).

\subsubsection*{Résultats}
Nous calculons le speedup séparément pour les deux séries (avec/sans L2), car les bases ($cycles_{app}(1)$) ne sont pas identiques.

\paragraph{Série A --- A7 avec L2 (base : $cycles_{app}(1)=2\,092\,404$).}

\scriptsize
\setlength{\tabcolsep}{6pt}
\renewcommand{\arraystretch}{1.20}
\begin{table}[H]
\centering
\caption{Speedup vs 1 thread --- A7 avec L2}
\label{tab:q6_speedup_l2}
\begin{tabular}{@{}r r r@{}}
\toprule
\textbf{Threads ($T$)} & \textbf{$cycles_{app}(T)$} & \textbf{Speedup$(T)$} \\
\midrule
1 & 2\,092\,404 & 1.000 \\
2 & 1\,128\,158 & 1.855 \\
4 & \ \ 646\,916 & 3.234 \\
8 & \ \ 408\,798 & 5.118 \\
\bottomrule
\end{tabular}
\end{table}

\paragraph{Série B --- A7 sans L2 (base : $cycles_{app}(1)=2\,597\,586$).}

\begin{table}[H]
\centering
\caption{Speedup vs 1 thread --- A7 sans L2}
\label{tab:q6_speedup_nol2}
\begin{tabular}{@{}r r r@{}}
\toprule
\textbf{Threads ($T$)} & \textbf{$cycles_{app}(T)$} & \textbf{Speedup$(T)$} \\
\midrule
1  & 2\,597\,586 & 1.000 \\
2  & 1\,508\,970 & 1.721 \\
4  & \ \ 970\,012 & 2.678 \\
8  & \ \ 705\,494 & 3.682 \\
16 & \ \ 582\,578 & 4.459 \\
\bottomrule
\end{tabular}
\end{table}
\normalsize