\subsection{Q7 --- Valeur maximale de l’IPC (à partir de \texttt{sim\_insts})}

\begin{formal}
\textbf{Énoncé (Q7).}\\
En utilisant le nombre total d’instructions simulées, déterminez quelle est la valeur maximale de l’IPC pour chaque configuration ?
\end{formal}

\subsubsection*{Définition et calcul}
Pour chaque exécution, gem5 fournit dans \texttt{m5out/stats.txt} :
\begin{itemize}[leftmargin=*]
  \item \texttt{sim\_insts} : le nombre total d’instructions simulées,
  \item $cycles_{app}$ : les cycles d’exécution de l’application (définis en Q5 comme $\max_i(\texttt{system.cpu}\langle i\rangle\texttt{.numCycles})$, voir aussi Q4--Q5).
\end{itemize}
On calcule alors l’IPC global par :
\[
IPC(T) \;=\; \frac{\texttt{sim\_insts}(T)}{cycles_{app}(T)}.
\]

\subsubsection*{Résultats}
Comme précédemment, on reporte deux séries séparées (avec/sans L2).

\paragraph{Série A --- A7 avec L2 ($T \le 8$).}

\scriptsize
\setlength{\tabcolsep}{5pt}
\renewcommand{\arraystretch}{1.20}
\begin{table}[H]
\centering
\caption{IPC --- A7 avec L2}
\label{tab:q7_ipc_l2}
\begin{tabular}{@{}r r r r@{}}
\toprule
\textbf{Threads} & \textbf{$cycles_{app}$} & \textbf{\texttt{sim\_insts}} & \textbf{IPC} \\
\midrule
1 & 2\,092\,404 & 4\,107\,655 & 1.963127 \\
2 & 1\,128\,158 & 4\,132\,898 & 3.663404 \\
4 & \ \ 646\,916 & 4\,158\,816 & 6.428680 \\
8 & \ \ 408\,798 & 4\,216\,901 & 10.315366 \\
\bottomrule
\end{tabular}
\end{table}

\noindent\textbf{IPC maximal (avec L2) :} $IPC_{max} = 10.315366$ (configuration $T=8$).

\paragraph{Série B --- A7 sans L2 ($T \le 16$).}

\begin{table}[H]
\centering
\caption{IPC --- A7 sans L2}
\label{tab:q7_ipc_nol2}
\begin{tabular}{@{}r r r r@{}}
\toprule
\textbf{Threads} & \textbf{$cycles_{app}$} & \textbf{\texttt{sim\_insts}} & \textbf{IPC} \\
\midrule
1  & 2\,597\,586 & 4\,107\,655 & 1.581336 \\
2  & 1\,508\,970 & 4\,233\,827 & 2.805773 \\
4  & \ \ 970\,012 & 4\,369\,018 & 4.504087 \\
8  & \ \ 705\,494 & 4\,570\,580 & 6.478553 \\
16 & \ \ 582\,578 & 5\,008\,208 & 8.596631 \\
\bottomrule
\end{tabular}
\end{table}
\normalsize

\noindent\textbf{IPC maximal (sans L2) :} $IPC_{max} = 8.596631$ (configuration $T=16$).

\subsubsection*{Observation courte}
Dans nos mesures, l’IPC global augmente avec le nombre de threads car $cycles_{app}$ diminue fortement lorsque l’on parallélise l’exécution. Les résultats sont reportés séparément pour les séries avec L2 et sans L2.