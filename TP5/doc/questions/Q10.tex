\subsection{Q10 --- Speedup par rapport à la configuration à 1 thread}

\begin{formal}
\textbf{Énoncé (Q10).}\\
Déduire le speedup par rapport à la configuration à 1 thread.
\end{formal}

\textbf{Réponse (Q10).}\\

Sachant que la formule de speedup est donnée par :
\[
\text{Speedup} = \frac{T_1}{T_n}
\]
où \( T_1 \) est le temps d'exécution de l'application avec 1 thread, et \( T_n \) est le temps d'exécution de l'application avec \( n \) threads, nous avons calculé le speedup pour chaque configuration testée.

Les résultats obtenus sont présentés dans le tableau suivant, où chaque ligne correspond à une configuration spécifique. Les colonnes indiquent le nombre de threads utilisés, le temps d'exécution en secondes, le speedup calculé par rapport à la configuration à 1 thread, et le nombre de cycles.

\begin{table}[h]
    \centering
    \caption{Speedup et cycles par rapport à la configuration à 1 thread}
    \begin{tabular}{cccccc}
        \hline\hline
        \textbf{Threads} & \textbf{Time (s)} & \textbf{Speedup} & \textbf{Cycles} \\
        \hline\hline
        1 & 0.004146 & 1.00 & 8130212000 \\
        2 & 0.002247 & 1.85 & 4146340500 \\
        4 & 0.001489 & 2.78 & 2246677000 \\
        8 & 0.001403 & 2.95 & 1601489500 \\
        \hline
    \end{tabular}
    \label{tab:speedup_cycles}
\end{table}
