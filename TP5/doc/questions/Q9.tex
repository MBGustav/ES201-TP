\subsection{Q9 --- Nombre de cycles d'exécution de l'application}
\setlength{\headheight}{13.6pt}

\begin{formal}
\textbf{Énoncé (Q9).}\\
Pour chaque configuration, quel est le nombre de cycles d'exécution de l'application? Vous pourrez présenter vos résultats sous forme de graphe 3 axes
\end{formal}

\textbf{Réponse (Q9).}\\
Pour répondre à cette question, nous avons mesuré le nombre de cycles d'exécution de l'application pour différentes configurations. Les configurations testées incluent des variations dans les paramètres suivants:
\begin{itemize}
    \item Le nombre de threads utilisés
    \item La taille des données traitées
    \item Le type d'algorithme utilisé
\end{itemize}

Les résultats obtenus sont présentés dans les tableaux suivants, où chaque ligne correspond à une configuration spécifique. Les colonnes indiquent le nombre de cycles d'exécution, le nombre d'instructions exécutées, et le temps de simulation en secondes.


\begin{table}[htp]
    \centering
    \caption{Cycles d'exécution pour 2 threads}
    \begin{tabular}{cccccc}
        \hline
        \textbf{Width} & \textbf{IPC Max CPU} & \textbf{Cycles Max CPU} & \textbf{Insts Max CPU} & \textbf{Sim Ticks} & \textbf{Sim Seconds} \\
        \hline\hline
        2 & 1.871782 & 8292682 & 15166137 & 4146340500 & 0.004146 \\
        2 & 1.807081 & 4493355 & 7789100 & 2246677000 & 0.002247 \\
        2 & 0.468555 & 2918550 & 1338205 & 1459274500 & 0.001459 \\
        2 & 0.314265 & 2805332 & 881618 & 1402665500 & 0.001403 \\
        \hline
    \end{tabular}
    \label{tab:cycles_execution2}
\end{table}


\begin{table}[htp]
    \centering
    \caption{Cycles d'exécution pour 4 threads}
    \begin{tabular}{ccccccc}
        \hline\hline
        \textbf{Width} & \textbf{IPC Max CPU} & \textbf{Cycles Max CPU} & \textbf{Insts Max CPU} & \textbf{Sim Ticks} & \textbf{Sim Seconds} \\
        \hline\hline
        4 & 2.218795 & 7058008 & 15166624 & 3529003500 & 0.003529 \\
        4 & 1.977961 & 4139440 & 7788713 & 2069719500 & 0.002070 \\
        4 & 1.437325 & 3202980 & 4101352 & 1601489500 & 0.001601 \\
        4 & 0.849159 & 2978819 & 2258180 & 1489409000 & 0.001489 \\
        \hline
    \end{tabular}
    \label{tab:cycles_execution4}
\end{table}

\begin{table}[htp]
    \centering
    \caption{Cycles d'exécution pour 8 threads}
    \begin{tabular}{ccccccc}
        \hline\hline
        \textbf{Width} & \textbf{IPC Max CPU} & \textbf{Cycles Max CPU} & \textbf{Insts Max CPU} & \textbf{Sim Ticks} & \textbf{Sim Seconds} \\
        \hline\hline
        8 & 2.244788 & 6980957 & 15166624 & 3490478000 & 0.003490 \\
        8 & 1.979324 & 4136693 & 7788720 & 2068346000 & 0.002068 \\
        8 & 1.476444 & 3211012 & 4142778 & 1605505500 & 0.001606 \\
        8 & 0.895454 & 2979155 & 2306598 & 1489577000 & 0.001490 \\
        \hline
    \end{tabular}
    \label{tab:cycles_execution8}
\end{table}

Et enfin, etant données les resultats obtenus, nous pouvons tracer un graphe à trois axes pour visualiser l'impact de ces différentes configurations sur le nombre de cycles d'exécution. La Figure~\ref{fig:cycles_execution_graph} montre les cycles d'exécution en fonction du nombre de threads et de la taille des données traitées.


\begin{figure}[htp]
    \centering
    \includegraphics[width=0.8\textwidth]{imgs/cycles_execution_graph.png}
    \caption{Graphe des cycles d'exécution en fonction du nombre de threads et de la taille des données}
    \label{fig:cycles_execution_graph}
\end{figure}


