\documentclass[11pt,a4paper]{article}

% ----------------------------------------------------
% PACKAGES ESSENTIELS
% ----------------------------------------------------
\usepackage[T1]{fontenc}
\usepackage[french]{babel}
\usepackage{geometry}
\usepackage{graphicx}
\usepackage{amsmath, amssymb}
\usepackage{siunitx}
\usepackage{caption}
\usepackage{subcaption}
\usepackage{booktabs}
\usepackage{float}
\usepackage{xcolor}
\usepackage{fancyhdr}
\usepackage{enumitem}
\usepackage{physics}
\usepackage{titlesec}
\usepackage[strict]{changepage}
\usepackage{framed}
\usepackage{hyperref}
\usepackage{tabularx}
\sisetup{round-mode=places, round-precision=6}

% ----------------------------------------------------
% PARAMÈTRES DE MISE EN PAGE
% ----------------------------------------------------
\geometry{margin=2.5cm}
\setlength{\parskip}{0.5em}
\setlength{\parindent}{0pt}

% ----------------------------------------------------
% COULEURS ENSTA
% ----------------------------------------------------
\definecolor{enstaBleuFonce}{HTML}{003366}
\definecolor{enstaBleuClair}{HTML}{0073CF}
\definecolor{formalshade}{rgb}{0.95,0.95,1}

% ----------------------------------------------------
% HYPERREF
% ----------------------------------------------------
\hypersetup{
    colorlinks=true,
    linkcolor=enstaBleuFonce,
    urlcolor=blue,
    citecolor=gray
}

% ----------------------------------------------------
% STYLE DES SECTIONS
% ----------------------------------------------------
\titleformat{\section}[block]
  {\normalfont\Large\bfseries\color{enstaBleuFonce}}
  {\thesection}{1em}{}
  [\vspace{0.3em}\titlerule\color{enstaBleuFonce}\vspace{0.3em}]

\titleformat{\subsection}
  {\normalfont\large\bfseries\color{enstaBleuClair}}
  {\thesubsection}{1em}{}

\titleformat{\subsubsection}
  {\normalfont\normalsize\bfseries\color{black!70}}
  {\thesubsubsection}{1em}{}

% ----------------------------------------------------
% EN-TÊTES ET PIEDS DE PAGE
% ----------------------------------------------------
\pagestyle{fancy}
\fancyhf{}
\fancyhead[L]{ENSTA Paris}
\fancyhead[R]{ECE\_4ES01\_TA / TP4 : Microprocesseur superscalaire \& mémoires caches}
\fancyfoot[C]{\thepage}

% ----------------------------------------------------
% ENVIRONNEMENT FORMAL (ENCADRÉ)
% ----------------------------------------------------
\newenvironment{formal}{%
\def\FrameCommand{%
  \hspace{1pt}%
  {\color{enstaBleuFonce}\vrule width 2pt}%
  {\color{formalshade}\vrule width 4pt}%
  \colorbox{formalshade}%
}%
\MakeFramed{\advance\hsize-\width\FrameRestore}%
\noindent\hspace{-4.55pt}%
\begin{adjustwidth}{}{7pt}%
\vspace{4pt}%
}{%
\vspace{4pt}\end{adjustwidth}\endMakeFramed%
}

% ----------------------------------------------------
% DÉBUT DU DOCUMENT
% ----------------------------------------------------
\begin{document}

% ====================================================
% PAGE DE COUVERTURE
% ====================================================
\begin{titlepage}
    \centering
    \vspace*{3.5cm}

    \includegraphics[width=0.55\textwidth]{imgs/logo_ensta_2025.png}

    {\Large Architecture des microprocesseurs (ECE\_4ES01\_TA) \par}
    \vspace{0.2cm}
    {\huge\bfseries TD/TP4 : Analyse de configurations d’architectures \par}
    \vspace{0.2cm}
    {\Large Microprocesseur superscalaire / mémoires caches \par}

    \vspace{2.5cm}

    % ====== AUTEURS (4) ======
    {\Large \textbf{Auteurs (quadrinôme)}\par}
    \vspace{0.4cm}
    {\Large CHACÓN GÓMEZ José Daniel \par}
    {\Large NOM Prénom 2 \par}
    {\Large NOM Prénom 3 \par}
    {\Large NOM Prénom 4 \par}

    \vspace{1.5cm}
    \textbf{Encadrant / Chargé de TD :} \underline{\hspace{6cm}} \par

    \vfill
    École Nationale des Techniques Avancées (ENSTA Paris)\\
    Février 2026 \par
\end{titlepage}

% ====================================================
% TABLE DES MATIÈRES
% ====================================================
\newpage
\tableofcontents
\newpage

% ====================================================
% CONTENU MODULAIRE
% ====================================================
\section{Introduction}
Dans ce TP, nous étudions l’impact de choix micro-architecturaux et de la hiérarchie mémoire
sur les performances, la surface et l’énergie, en nous appuyant sur des simulations (gem5) et sur
les applications \texttt{Dijkstra} et \texttt{BlowFish}.

\section{Exercice 3 --- A completer}

\subsection{Questions}
\subsubsection{Question 1 - Configuration de cache}

\begin{formal}
\textbf{Énoncé (Q1).}
Déterminez, pour chacune des deux configurations de mémoires cache, les
paramètres de configuration des caches à utiliser dans gem5 et complétez la Table~\ref{tab:cache_conf}.
\end{formal}
Dans cette question, nous allons comparer les performances de deux configurations de cache différentes, C1 et C2. Les caractéristiques de chaque configuration sont présentées dans le tableau ci-dessous.


\begin{lstlisting}[language=Python, caption=Extrait du fichier \texttt{se\_cache.py}, label={lst:cache_conf}]

def apply_cache_conf(args, system):
    """
    Cree I$, D$, L2 selon C1/C2 ou selon parametres custom.
    """
    system.cache_line_size = args.line_size

    # Selection config

    if args.conf == "C1":
        l1i_size, l1i_assoc = "4kB", 1
        l1d_size, l1d_assoc = "4kB", 1
        l2_size,  l2_assoc  = "32kB", 1
    elif args.conf == "C2":
        l1i_size, l1i_assoc = "4kB", 1
        l1d_size, l1d_assoc = "4kB", 2
        l2_size,  l2_assoc  = "32kB", 4
    
    # L1
    system.cpu.icache = L1ICache()
    system.cpu.dcache = L1DCache()
    system.cpu.icache.size = l1i_size
    system.cpu.icache.assoc = l1i_assoc
    system.cpu.dcache.size = l1d_size
    system.cpu.dcache.assoc = l1d_assoc
    
    # L2
    system.l2cache = L2Cache()
    #(...)
    # Mem bus
    system.membus = SystemXBar()
    #(...)

\end{lstlisting}

Dans le extrait presenté dans le Listing~\ref{lst:cache_conf}, nous pouvons voir les configurations de cache C1 et C2 qui sont appliquées. Il faut considerer que la configuration C1 correspond à une architecture de cache plus simple, tandis que la configuration C2 apresente une diference à ce qui concerne a une associativité pour le cache de données (DL2) et le cache de niveau 2 (UL2).



\begin{table}[htp]

    \centering
    \begin{tabular}{|c|ccc}
        Configuration & IL1 & DL2 & UL2 \\\hline\hline
        
        C1 & size=4kB, assoc=1 & size=4kB, assoc=1 & size=32kB, assoc=1 \\
        C2 & size=4kB, assoc=1 & size=4kB, assoc=1 & size=32kB, assoc=4 \\    
    \end{tabular}
    \caption{Parametres de cache pour les configurations C1 et C2}

\end{table}


\subsubsection{Question 2 - Comparaison de perrformances}

\begin{formal}
\textbf{Énoncé (Q2).}
Examinez le fichier de déclaration d’un élément de type « processeur superscalaire out-of-order », et présentez sous forme de tableau cinq paramètres configurables de ce type de processeur avec leur valeur par défaut. Choisissez de préférence des paramètres étudiés lors des séances TD/TP précédentes.  Le fichier à consulter est le suivant: \$GEM5/src/cpu/o3/O3CPU.py
\end{formal}

\begin{table}
\centering
\begin{tabular}{|c|c|c|} 
    \textbf{Paramètre} & \textbf{Description} & \textbf{Valeur par défaut} \\\hline\hline
    fetchWidth    & Nº d'instr. que peut récupérer par cycle & 4 \\
    decodeWidth   & Nº d'instr. que peut décoder par cycle & 4 \\
    issueWidth    & Nº d'instr. que peut émettre par cycle & 4 \\
    commitWidth   & Nº d'instr. que peut valider par cycle & 4 \\
    numROBEntries & Nº d'entrées dans la Reorder Buffer (ROB) & 192 \\
\end{tabular}
\caption{Paramètres configurables d'un processeur superscalaire out-of-order}
\label{tab:params}

\end{table}
\subsubsection{Qi}
Contenu de l'Exercice 3 -- Qi a completer.


\section{Exercice 4 --- Architecture big.LITTLE : Dijkstra \& BlowFish}

\subsection{1. Profiling de l'application}
\subsubsection{Question 1 - Configuration de cache}

\begin{formal}
\textbf{Énoncé (Q1).}
Déterminez, pour chacune des deux configurations de mémoires cache, les
paramètres de configuration des caches à utiliser dans gem5 et complétez la Table~\ref{tab:cache_conf}.
\end{formal}
Dans cette question, nous allons comparer les performances de deux configurations de cache différentes, C1 et C2. Les caractéristiques de chaque configuration sont présentées dans le tableau ci-dessous.


\begin{lstlisting}[language=Python, caption=Extrait du fichier \texttt{se\_cache.py}, label={lst:cache_conf}]

def apply_cache_conf(args, system):
    """
    Cree I$, D$, L2 selon C1/C2 ou selon parametres custom.
    """
    system.cache_line_size = args.line_size

    # Selection config

    if args.conf == "C1":
        l1i_size, l1i_assoc = "4kB", 1
        l1d_size, l1d_assoc = "4kB", 1
        l2_size,  l2_assoc  = "32kB", 1
    elif args.conf == "C2":
        l1i_size, l1i_assoc = "4kB", 1
        l1d_size, l1d_assoc = "4kB", 2
        l2_size,  l2_assoc  = "32kB", 4
    
    # L1
    system.cpu.icache = L1ICache()
    system.cpu.dcache = L1DCache()
    system.cpu.icache.size = l1i_size
    system.cpu.icache.assoc = l1i_assoc
    system.cpu.dcache.size = l1d_size
    system.cpu.dcache.assoc = l1d_assoc
    
    # L2
    system.l2cache = L2Cache()
    #(...)
    # Mem bus
    system.membus = SystemXBar()
    #(...)

\end{lstlisting}

Dans le extrait presenté dans le Listing~\ref{lst:cache_conf}, nous pouvons voir les configurations de cache C1 et C2 qui sont appliquées. Il faut considerer que la configuration C1 correspond à une architecture de cache plus simple, tandis que la configuration C2 apresente une diference à ce qui concerne a une associativité pour le cache de données (DL2) et le cache de niveau 2 (UL2).



\begin{table}[htp]

    \centering
    \begin{tabular}{|c|ccc}
        Configuration & IL1 & DL2 & UL2 \\\hline\hline
        
        C1 & size=4kB, assoc=1 & size=4kB, assoc=1 & size=32kB, assoc=1 \\
        C2 & size=4kB, assoc=1 & size=4kB, assoc=1 & size=32kB, assoc=4 \\    
    \end{tabular}
    \caption{Parametres de cache pour les configurations C1 et C2}

\end{table}


\subsubsection{Question 2 - Comparaison de perrformances}

\begin{formal}
\textbf{Énoncé (Q2).}
Examinez le fichier de déclaration d’un élément de type « processeur superscalaire out-of-order », et présentez sous forme de tableau cinq paramètres configurables de ce type de processeur avec leur valeur par défaut. Choisissez de préférence des paramètres étudiés lors des séances TD/TP précédentes.  Le fichier à consulter est le suivant: \$GEM5/src/cpu/o3/O3CPU.py
\end{formal}

\begin{table}
\centering
\begin{tabular}{|c|c|c|} 
    \textbf{Paramètre} & \textbf{Description} & \textbf{Valeur par défaut} \\\hline\hline
    fetchWidth    & Nº d'instr. que peut récupérer par cycle & 4 \\
    decodeWidth   & Nº d'instr. que peut décoder par cycle & 4 \\
    issueWidth    & Nº d'instr. que peut émettre par cycle & 4 \\
    commitWidth   & Nº d'instr. que peut valider par cycle & 4 \\
    numROBEntries & Nº d'entrées dans la Reorder Buffer (ROB) & 192 \\
\end{tabular}
\caption{Paramètres configurables d'un processeur superscalaire out-of-order}
\label{tab:params}

\end{table}

\subsection{2. Evaluation des performances}
\subsubsection{Question 4 - Localité de références}
\begin{formal}
    \textbf{Question 4:} Déterminez quel est le processeur exécutant toujours le plus grand nombre de cycles. Expliquez pourquoi. expliquez également pourquoi l'analyse du nombre de cycles sur ce processeur revient à analyser le nombre total de cycles d'exécution de l'application.
\end{formal}




\subsubsection{Q5 — Cortex A15 : impact de la taille de L1 (L1I = L1D)}

\begin{formal}
\textbf{Énoncé (Q5).}
Générez les figures de performances détaillées (performance générale, IPC, hiérarchie mémoire,
prédiction de branchement, etc.) en fonction de la taille du cache L1 pour les configurations testées.
Analysez les résultats. Quelle configuration de L1 donne les meilleures performances pour le Cortex A15
pour \texttt{dijkstra} ? et pour \texttt{BlowFish} ?
\newline\textbf{N.B. : Mentionnez les paramètres d’exécution de Gem5 que vous avez utilisé.}
\end{formal}

\paragraph{Paramètres d’exécution (gem5).}
\begin{itemize}[leftmargin=*, itemsep=0.2em]
  \item Gem5 (mode SE, ISA RISC-V) : \texttt{build/RISCV/gem5.opt}
  \item Script de configuration : \texttt{TP4/se\_A15.py} (CPU OoO)
  \item Paramètres CLI utilisés : \texttt{-d <outdir>}, \texttt{--cmd <binary>}, \texttt{--options <args>},
        \texttt{--l1i-size <NkB>}, \texttt{--l1d-size <NkB>}
  \item Balayage : \texttt{--l1i-size = --l1d-size} $\in \{2,4,8,16,32\}\,\text{kB}$
  \item L2 fixé à 512kB dans le script.
\end{itemize}

\begin{table}[H]
\centering
\caption{Q5 (Cortex A15) — \texttt{dijkstra\_large} : métriques vs taille L1 (L1I=L1D, L2=512kB)}
\label{tab:q5_a15_dijkstra}
\scriptsize
\setlength{\tabcolsep}{4.5pt}
\renewcommand{\arraystretch}{1.10}
\begin{tabular}{c c c c c c c c}
\toprule
\textbf{L1 (kB)} & \textbf{IPC} & \textbf{CPI} & \textbf{Cycles (M)} &
\textbf{I\$ miss} & \textbf{D\$ miss} & \textbf{L2 miss} & \textbf{Mispred} \\
\midrule
 2  & 0.654236 & 1.528500 & 311.682 & 0.052157 & 0.185632 & 0.000220 & 0.011167 \\
 4  & 0.716546 & 1.395584 & 284.579 & 0.026767 & 0.143118 & 0.000290 & 0.011140 \\
 8  & 0.901382 & 1.109407 & 226.223 & 0.005627 & 0.077506 & 0.000574 & 0.011522 \\
 16 & 0.981770 & 1.018568 & 207.700 & 0.002244 & 0.051970 & 0.000871 & 0.011546 \\
 32 & 1.141754 & 0.875845 & 178.597 & 0.001183 & 0.017590 & 0.002893 & 0.011588 \\
\bottomrule
\end{tabular}
\end{table}

\begin{table}[H]
\centering
\caption{Q5 (Cortex A15) — \texttt{blowfish\_large} : métriques vs taille L1 (L1I=L1D, L2=512kB)}
\label{tab:q5_a15_blowfish}
\scriptsize
\setlength{\tabcolsep}{4.5pt}
\renewcommand{\arraystretch}{1.10}
\begin{tabular}{c c c c c c c c}
\toprule
\textbf{L1 (kB)} & \textbf{IPC} & \textbf{CPI} & \textbf{Cycles (M)} &
\textbf{I\$ miss} & \textbf{D\$ miss} & \textbf{L2 miss} & \textbf{Mispred} \\
\midrule
 2  & 1.060149 & 0.943264 & 12.441 & 0.110876 & 0.176653 & 0.002642 & 0.125675 \\
 4  & 1.134343 & 0.881568 & 11.627 & 0.019981 & 0.124223 & 0.004438 & 0.124880 \\
 8  & 1.359661 & 0.735477 &  9.701 & 0.000680 & 0.034137 & 0.022683 & 0.124722 \\
 16 & 1.390119 & 0.719363 &  9.488 & 0.000507 & 0.029321 & 0.030426 & 0.124713 \\
 32 & 1.497467 & 0.667794 &  8.808 & 0.000448 & 0.001067 & 0.798194 & 0.124786 \\
\bottomrule
\end{tabular}
\end{table}

\begin{figure}[H]
\centering
\includegraphics[width=0.95\linewidth]{plot_q5_a15_dijkstra_large.png}
\caption{Q5 (A15) — \texttt{dijkstra\_large} : IPC, cycles, hiérarchie mémoire et prédiction de branchement vs taille L1.}
\label{fig:q5_a15_dijkstra}
\end{figure}

\begin{figure}[H]
\centering
\includegraphics[width=0.95\linewidth]{plot_q5_a15_blowfish_large.png}
\caption{Q5 (A15) — \texttt{blowfish\_large} : IPC, cycles, hiérarchie mémoire et prédiction de branchement vs taille L1.}
\label{fig:q5_a15_blowfish}
\end{figure}

\paragraph{Analyse}
Les résultats des Tables~\ref{tab:q5_a15_dijkstra} et~\ref{tab:q5_a15_blowfish}, confirmés par les Figures~\ref{fig:q5_a15_dijkstra}
et~\ref{fig:q5_a15_blowfish}, montrent une tendance nette : quand la taille de L1 augmente (\textbf{L1I=L1D}),
\textbf{I\$ miss} et \textbf{D\$ miss} chutent, ce qui se traduit par \textbf{moins de cycles} et \textbf{un IPC plus élevé}.
Cela a du sens, car réduire les misses réduit les attentes vers L2/mémoire et donc les \emph{stalls}.

Pour \texttt{dijkstra\_large} (Table~\ref{tab:q5_a15_dijkstra}, Figure~\ref{fig:q5_a15_dijkstra}), l’amélioration est régulière :
on observe simultanément la baisse des misses (surtout \textbf{D\$}) et la hausse de l’IPC, ce qui indique que l’application
bénéficie directement d’un meilleur taux de hits en L1.

Pour \texttt{blowfish\_large} (Table~\ref{tab:q5_a15_blowfish}, Figure~\ref{fig:q5_a15_blowfish}), le gain est surtout marqué
entre {2}{kB} et {8}{kB} : les misses chutent très rapidement et l’IPC augmente fortement, ce qui correspond bien à une charge
\emph{bouclée} (code répétitif + tables) dont le \emph{working set} tient progressivement en cache.
Le taux de misprediction reste globalement stable (Figures~\ref{fig:q5_a15_dijkstra} et~\ref{fig:q5_a15_blowfish}),
donc la performance est principalement pilotée par la hiérarchie mémoire.

\paragraph{Remarque (BlowFish, L1={32}{kB}) : pic de miss L2.}
Sur la Figure~\ref{fig:q5_a15_blowfish} et la Table~\ref{tab:q5_a15_blowfish}, le \textbf{miss L2} augmente fortement à {32}{kB}.
D’après \texttt{stats.txt}, cela s’explique surtout par un \textbf{effet de ratio} : seulement \(m 2547\) accès atteignent L2,
et une grande partie sont des misses, alors même que \(D\$\) miss devient quasi nul (accès majoritairement servis en L1).
La tendance globale (cycles en baisse, IPC en hausse) confirme que l’effet dominant reste la chute des misses L1.

\paragraph{Meilleure configuration (A15).}
Sur les points testés, la meilleure performance est obtenue avec \textbf{L1I=L1D={32}{kB}} pour \texttt{dijkstra} et \texttt{BlowFish}
(min cycles / max IPC).


\subsection{3. Efficacité surfacique}
\subsubsection{Q6 — Paramètres par défaut dans \texttt{cache.cfg}}

\begin{formal}
\textbf{Énoncé (Q6).}
Observez le fichier de configuration \texttt{cache.cfg}. Quels sont les paramètres de cache
(taille, taille de bloc, associativité) et la technologie (nm) utilisés par défaut ?
\end{formal}

\paragraph{Réponse.}
D’après les lignes non commentées du fichier \texttt{cache.cfg}, la configuration par défaut est :
\begin{itemize}[leftmargin=*, itemsep=0.2em]
  \item Taille du cache : \textbf{32 kB} (\texttt{-size (bytes) 32768})
  \item Taille de bloc (ligne) : \textbf{64 B} (\texttt{-block size (bytes) 64})
  \item Associativité : \textbf{2-way} (\texttt{-associativity 2})
  \item Technologie : \textbf{32 nm} (\texttt{-technology (u) 0.032})
\end{itemize}

\paragraph{Remarque (technologie).}
Le sujet cible 28 nm, mais la version de CACTI utilisée dans ce TP ne permet pas d’obtenir un résultat stable à 28 nm
(abandon interne lors de l’exécution). Conformément à la note de l’énoncé (32 nm si 28 nm n’est pas supporté),
les estimations de surface sont donc réalisées en 32 nm (\texttt{-technology (u) 0.032}, soit 0.032 um).

\subsubsection{Q7 — Surface L1, pourcentage et taille des coeurs}

\begin{formal}
\textbf{Énoncé (Q7).}
Quelle est la surface des caches L1 du Tableau 12 (instructions et données) en mm$^2$ ?
Quel pourcentage de la surface totale des coeurs A7 et A15 est occupé par les caches L1 ?
En déduire la taille des deux coeurs (hors caches L1). Donnez votre analyse.
\end{formal}

\paragraph{Méthode.}
Nous utilisons CACTI à 32 nm (cf. Q6), avec les paramètres L1 du Tableau 12 :
\begin{itemize}[leftmargin=*, itemsep=0.2em]
  \item A7 : 32 kB, bloc 32 B, associativité 2
  \item A15 : 32 kB, bloc 64 B, associativité 2
\end{itemize}
Les fichiers \texttt{cache\_L1\_A7\_32nm.cfg} et \texttt{cache\_L1\_A15\_32nm.cfg}
ont été créés à partir de \texttt{cache.cfg}, en modifiant uniquement les paramètres L1 utiles.
Pour Q7, on utilise seulement les champs du Tableau 12 suivants : \textbf{I-L1\$} et \textbf{D-L1\$}
(taille, taille de bloc, associativité).

Dans chaque sortie CACTI, la surface d'un cache L1 est calculée par :
\[
S_{L1} = S_{\text{data}} + S_{\text{tag}}
\]
Puis :
\[
S_{L1,\text{total}} = S_{L1I} + S_{L1D},\quad
\%L1 = \frac{S_{L1,\text{total}}}{S_{\text{coeur+L1}}}\times 100,\quad
S_{\text{coeur seul}} = S_{\text{coeur+L1}} - S_{L1,\text{total}}
\]
avec \(S_{\text{coeur+L1}}=0.45\) mm$^2$ pour A7 et \(2.0\) mm$^2$ pour A15 (énoncé).

\begin{table}[H]
\centering
\caption{Q7 — Résultats de surface (CACTI 32nm)}
\label{tab:q7_surface}
\small
\setlength{\tabcolsep}{4.5pt}
\renewcommand{\arraystretch}{1.10}
\begin{tabular}{l c c}
\toprule
\textbf{Métrique} & \textbf{A7} & \textbf{A15} \\
\midrule
\(S_{\text{data}}\) d'un L1 (mm$^2$) & 0.0301 & 0.0301 \\
\(S_{\text{tag}}\) d'un L1 (mm$^2$) & 0.0083 & 0.0044 \\
\(S_{L1I}\) (mm$^2$) & 0.0384 & 0.0346 \\
\(S_{L1D}\) (mm$^2$) & 0.0384 & 0.0346 \\
\(S_{L1,\text{total}}\) (mm$^2$) & 0.0769 & 0.0691 \\
\(\%L1\) de la surface coeur+L1 & 17.08\% & 3.46\% \\
\(S_{\text{coeur seul}}\) (mm$^2$) & 0.3731 & 1.9309 \\
\bottomrule
\end{tabular}
\end{table}

\paragraph{Analyse.}
Les L1 représentent une part nettement plus importante sur A7 (\(\approx 17.1\%\)) que sur A15 (\(\approx 3.46\%\)).
Autrement dit, la surface ``logique coeur'' (hors L1) est dominante sur A15, ce qui est cohérent avec un coeur plus large
et plus agressif micro-architecturalement. À paramètres L1 comparables (32 kB, 2-way), la différence principale
provient ici de la taille de bloc (32 B vs 64 B), qui modifie le coût de la partie tag.

% !TeX root = ../../rapport.tex
\subsubsection{Q8 --- Variation de la taille L1 et nouvelle surface totale}

\begin{formal}
\textbf{Énoncé (Q8).}
Faire varier la taille des caches L1 pour le Cortex A7 et le Cortex A15, puis donner (en mm$^2$) les surfaces L1 obtenues.
Avec la configuration L2 utilisée précédemment (512 kB), en déduire pour chaque valeur les nouvelles surfaces totales
(coeur + L1 + L2), et présenter les résultats sous forme de graphes.
\end{formal}

\paragraph{Méthode.}
Pour chaque architecture, nous avons fait un balayage des tailles L1 (L1I = L1D) avec CACTI en 32 nm.
Les paramètres de blocs/associativité suivent le Tableau 12 :
\begin{itemize}[leftmargin=*, itemsep=0.2em]
  \item A7 : taille L1 variable (1, 2, 4, 8, 16 kB), bloc 32 B, associativité 2 ; L2 = 512 kB, bloc 32 B, associativité 8
  \item A15 : taille L1 variable (2, 4, 8, 16, 32 kB), bloc 64 B, associativité 2 ; L2 = 512 kB, bloc 64 B, associativité 16
\end{itemize}
Les calculs utilisés sont :
\[
S_{L1,\text{total}} = S_{L1I} + S_{L1D},\qquad
S_{\text{total}} = S_{\text{coeur sans L1}} + S_{L1,\text{total}} + S_{L2}
\]
avec les surfaces coeur sans L1 obtenues en Q7.

\begin{table}[H]
\centering
\caption{Q8 --- A7 : surface L1 totale et surface totale (coeur + L1 + L2)}
\label{tab:q8_a7}
\small
\setlength{\tabcolsep}{7pt}
\renewcommand{\arraystretch}{1.10}
\begin{tabular}{c c c}
\toprule
\textbf{L1 (kB)} & \textbf{$S_{L1,\text{total}}$ (mm$^2$)} & \textbf{$S_{\text{total}}$ (mm$^2$)} \\
\midrule
1  & 0.0085 & 0.8274 \\
2  & 0.0209 & 0.8397 \\
4  & 0.0114 & 0.8302 \\
8  & 0.0257 & 0.8445 \\
16 & 0.0324 & 0.8513 \\
\bottomrule
\end{tabular}
\end{table}

\begin{table}[H]
\centering
\caption{Q8 --- A15 : surface L1 totale et surface totale (coeur + L1 + L2)}
\label{tab:q8_a15}
\small
\setlength{\tabcolsep}{7pt}
\renewcommand{\arraystretch}{1.10}
\begin{tabular}{c c c}
\toprule
\textbf{L1 (kB)} & \textbf{$S_{L1,\text{total}}$ (mm$^2$)} & \textbf{$S_{\text{total}}$ (mm$^2$)} \\
\midrule
2  & 0.0202 & 2.3506 \\
4  & 0.0102 & 2.3406 \\
8  & 0.0236 & 2.3539 \\
16 & 0.0282 & 2.3586 \\
32 & 0.0691 & 2.3995 \\
\bottomrule
\end{tabular}
\end{table}

\begin{figure}[H]
\centering
\includegraphics[width=0.88\linewidth]{plot_q8_l1_area_vs_size.png}
\caption{Q8 --- Surface L1 totale en fonction de la taille L1 (A7 et A15), en valeurs absolues (mm$^2$), pas en pourcentage.}
\label{fig:q8_l1_area}
\end{figure}

\begin{figure}[H]
\centering
\includegraphics[width=0.95\linewidth]{plot_q8_total_area_vs_size.png}
\caption{Q8 --- Surface totale en fonction de la taille L1 (gauche) et variation de surface par rapport a la plus petite L1 testée (droite).}
\label{fig:q8_total_area}
\end{figure}

\paragraph{Analyse.}
Les courbes de la Figure~\ref{fig:q8_l1_area} montrent que la surface des caches L1 augmente globalement quand la taille L1 augmente.
Ce résultat est logique : plus de capacité implique plus de cellules mémoire et donc plus de surface occupée.
Important : cette figure est une comparaison en mm$^2$ absolus (et non une comparaison en \%).
Donc, si deux points paraissent proches, cela ne veut pas dire que les deux architectures ont la même taille globale ;
cela veut seulement dire que la surface L1 estimée par CACTI est proche pour ces configurations.
% !TeX root = ../../rapport.tex
\subsubsection{Q9 --- Efficacite surfacique (IPC / mm$^2$) selon la taille de L1}

\begin{formal}
\textbf{Énoncé (Q9).}
En prenant en compte les deux dimensions (performance et surface) pour les deux processeurs considérés,
donnez pour chaque configuration de L1 l'efficacite surfacique de chaque processeur.
\newline\textbf{N.B. :} Efficacite surfacique = IPC / surface(mm$^2$).
\end{formal}

\paragraph{Méthode.}
Pour la performance, nous utilisons l'IPC mesuré sous gem5 :
\begin{itemize}[leftmargin=*, itemsep=0.2em]
  \item A7 : IPC issus de Q4 (tailles L1 = 1, 2, 4, 8, 16 kB)
  \item A15 : IPC issus de Q5 (tailles L1 = 2, 4, 8, 16, 32 kB)
\end{itemize}
Pour la surface, nous utilisons les surfaces totales estimées en Q8 avec CACTI :
coeur sans L1 + L1I + L1D + L2 (avec L1I = L1D dans notre balayage).
L2 est fixe a 512 kB (technologie 32 nm). Pour chaque point :
\[
\text{efficacite surfacique} = \frac{\text{IPC}}{\text{surface totale (mm$^2$)}}
\]

\begin{table}[H]
\centering
\caption{Q9 --- A7 : efficacite surfacique (IPC / mm$^2$) selon L1}
\label{tab:q9_a7}
\scriptsize
\setlength{\tabcolsep}{4.5pt}
\renewcommand{\arraystretch}{1.10}
\begin{tabular}{c c c c c c}
\toprule
\textbf{L1 (kB)} & \textbf{Surface totale (mm$^2$)} & \textbf{IPC (dijkstra)} & \textbf{IPC/mm$^2$ (dijkstra)} & \textbf{IPC (blowfish)} & \textbf{IPC/mm$^2$ (blowfish)} \\
\midrule
1  & 0.8274 & 0.2319 & 0.2802 & 0.2520 & 0.3045 \\
2  & 0.8397 & 0.2394 & 0.2851 & 0.2579 & 0.3072 \\
4  & 0.8302 & 0.2499 & 0.3010 & 0.2702 & 0.3254 \\
8  & 0.8445 & 0.2714 & 0.3214 & 0.2966 & 0.3512 \\
16 & 0.8513 & 0.2787 & 0.3274 & 0.2981 & 0.3502 \\
\bottomrule
\end{tabular}
\end{table}

\begin{table}[H]
\centering
\caption{Q9 --- A15 : efficacite surfacique (IPC / mm$^2$) selon L1}
\label{tab:q9_a15}
\scriptsize
\setlength{\tabcolsep}{4.5pt}
\renewcommand{\arraystretch}{1.10}
\begin{tabular}{c c c c c c}
\toprule
\textbf{L1 (kB)} & \textbf{Surface totale (mm$^2$)} & \textbf{IPC (dijkstra)} & \textbf{IPC/mm$^2$ (dijkstra)} & \textbf{IPC (blowfish)} & \textbf{IPC/mm$^2$ (blowfish)} \\
\midrule
2  & 2.3506 & 0.6542 & 0.2783 & 1.0601 & 0.4510 \\
4  & 2.3406 & 0.7165 & 0.3061 & 1.1343 & 0.4846 \\
8  & 2.3539 & 0.9014 & 0.3829 & 1.3597 & 0.5776 \\
16 & 2.3586 & 0.9818 & 0.4163 & 1.3901 & 0.5894 \\
32 & 2.3995 & 1.1418 & 0.4758 & 1.4975 & 0.6241 \\
\bottomrule
\end{tabular}
\end{table}

\begin{figure}[H]
\centering
\includegraphics[width=0.98\linewidth]{plot_q9_efficiency.png}
\caption{Q9 --- Efficacite surfacique (IPC / mm$^2$) en fonction de la taille L1, pour A7 (gauche) et A15 (droite).}
\label{fig:q9_eff}
\end{figure}

\paragraph{Analyse.}
La Figure~\ref{fig:q9_eff} et les Tables~\ref{tab:q9_a7} et~\ref{tab:q9_a15} montrent que l'efficacite surfacique
augmente globalement quand on augmente L1. C'est un resultat logique ici, car l'augmentation de surface totale entre deux points
reste relativement faible (la surface est dominée par le coeur sans L1 et par L2), alors que l'IPC beneficie directement
de la baisse des misses quand L1 grandit.

Pour A7, on observe clairement un effet de rendements decroissants : entre 4 kB et 8 kB, l'efficacite surfacique progresse nettement
(environ +6.8\% sur \texttt{dijkstra} et +7.9\% sur \texttt{blowfish}), alors qu'entre 8 kB et 16 kB le gain est faible
(environ +1.9\% sur \texttt{dijkstra} et quasi nul sur \texttt{blowfish}). Cela indique qu'a partir de 8 kB, une grande partie du
\emph{working set} critique tient deja dans L1 : augmenter davantage L1 apporte peu en IPC, donc peu en IPC/mm$^2$.
Ainsi, meme si 16 kB est le maximum pour \texttt{dijkstra}, 8 kB apparait comme un choix tres proche et plus \"equilibre\".

Pour A15, la tendance est plus favorable a de grandes L1 : le passage de 16 kB a 32 kB augmente l'efficacite surfacique de maniere
plus significative (environ +14.3\% sur \texttt{dijkstra} et +5.9\% sur \texttt{blowfish}), ce qui justifie que 32 kB soit la meilleure
configuration sur les points testes.


\subsection{4. Efficacité énergétique}
% !TeX root = ../../rapport.tex
\subsubsection{Q10 --- Puissance consommee a la frequence maximale}

\begin{formal}
\textbf{Énoncé (Q10).}
Quelle puissance en mW consomme chaque processeur a la frequence maximale ?
\end{formal}

\paragraph{Réponse.}
Les donnees du sujet indiquent une consommation de 0.10 mW/MHz (A7) et 0.20 mW/MHz (A15),
avec des frequences maximales de 1.0 GHz (A7) et 2.5 GHz (A15).
On convertit en MHz (1.0 GHz = 1000 MHz, 2.5 GHz = 2500 MHz) puis :
\[
P = \text{(mW/MHz)} \times f_{\max}(\text{MHz})
\]

\begin{itemize}[leftmargin=*, itemsep=0.2em]
  \item A7 : $0.10 \times 1000 = 100$ mW
  \item A15 : $0.20 \times 2500 = 500$ mW
\end{itemize}

% !TeX root = ../../rapport.tex
\subsubsection{Q11 --- Efficacite energetique (IPC / mW) selon la taille de L1}

\begin{formal}
\textbf{Énoncé (Q11).}
Avec le meme protocole que precedemment, et en prenant en compte les deux dimensions (energie et surface)
pour les deux processeurs consideres, donnez pour chaque configuration de L1 l'efficacite energetique de chaque processeur
(a frequence maximale).
\newline\textbf{N.B. :} Efficacite energetique = IPC / consommation energie (mW).
\end{formal}

\paragraph{Méthode.}
Nous utilisons l'IPC mesure sous gem5 (A7: Q4, A15: Q5) et la puissance a frequence maximale calculee en Q10.
Ici, la puissance est supposee constante pour un coeur donne (d'apres les donnees du sujet), donc :
\[
\text{efficacite energetique} = \frac{\text{IPC}}{P_{\max}(\text{mW})}
\]
avec $P_{\max}=100$ mW pour A7 et $P_{\max}=500$ mW pour A15.

\begin{table}[H]
\centering
\caption{Q11 --- A7 : efficacite energetique (IPC / mW) selon L1}
\label{tab:q11_a7}
\scriptsize
\setlength{\tabcolsep}{4.5pt}
\renewcommand{\arraystretch}{1.10}
\begin{tabular}{c c c c c}
\toprule
\textbf{L1 (kB)} & \textbf{IPC (dijkstra)} & \textbf{IPC/mW (dijkstra)} & \textbf{IPC (blowfish)} & \textbf{IPC/mW (blowfish)} \\
\midrule
1  & 0.2319 & 0.002319 & 0.2520 & 0.002520 \\
2  & 0.2394 & 0.002394 & 0.2579 & 0.002579 \\
4  & 0.2499 & 0.002499 & 0.2702 & 0.002702 \\
8  & 0.2714 & 0.002714 & 0.2966 & 0.002966 \\
16 & 0.2787 & 0.002787 & 0.2981 & 0.002981 \\
\bottomrule
\end{tabular}
\end{table}

\begin{table}[H]
\centering
\caption{Q11 --- A15 : efficacite energetique (IPC / mW) selon L1}
\label{tab:q11_a15}
\scriptsize
\setlength{\tabcolsep}{4.5pt}
\renewcommand{\arraystretch}{1.10}
\begin{tabular}{c c c c c}
\toprule
\textbf{L1 (kB)} & \textbf{IPC (dijkstra)} & \textbf{IPC/mW (dijkstra)} & \textbf{IPC (blowfish)} & \textbf{IPC/mW (blowfish)} \\
\midrule
2  & 0.6542 & 0.001308 & 1.0601 & 0.002120 \\
4  & 0.7165 & 0.001433 & 1.1343 & 0.002269 \\
8  & 0.9014 & 0.001803 & 1.3597 & 0.002719 \\
16 & 0.9818 & 0.001964 & 1.3901 & 0.002780 \\
32 & 1.1418 & 0.002284 & 1.4975 & 0.002995 \\
\bottomrule
\end{tabular}
\end{table}

\begin{figure}[H]
\centering
\includegraphics[width=0.98\linewidth]{plot_q11_efficiency.png}
\caption{Q11 --- Efficacite energetique (IPC / mW) en fonction de la taille L1, pour A7 (gauche) et A15 (droite).}
\label{fig:q11_eff}
\end{figure}

\paragraph{Analyse.}
La Figure~\ref{fig:q11_eff} et les Tables~\ref{tab:q11_a7} et~\ref{tab:q11_a15} donnent directement l'efficacite energetique
(IPC/mW) pour chaque taille de L1. Dans ce modele, $P_{\max}$ est constant par coeur, donc l'evolution d'IPC/mW suit
exactement l'evolution de l'IPC.


\subsection{5. Architecture système big.LITTLE}
% !TeX root = ../../rapport.tex
\subsubsection{Q12 --- Choix de la configuration L1 pour un systeme big.LITTLE}

\begin{formal}
\textbf{Énoncé (Q12).}
Avec un esprit de concepteur de systeme, et en se basant sur les resultats de Q10 et Q11,
proposez la meilleure configuration du cache L1 du processeur big.LITTLE pour les applications
\texttt{dijkstra} et \texttt{blowfish} individuellement.
\end{formal}

\paragraph{Proposition (par application).}
Les gains en \% ci-dessous sont des gains relatifs d'efficacite energetique (IPC/mW) entre deux tailles L1 comparees.
\begin{itemize}[leftmargin=*, itemsep=0.25em]
  \item \textbf{Application \texttt{dijkstra}} :
        \begin{itemize}[leftmargin=*, itemsep=0.15em]
          \item A7 : \textbf{16 kB} (meilleur IPC/mW sur les points testes, Table~\ref{tab:q11_a7}).
                Entre 4 kB $\rightarrow$ 8 kB, l'efficacite energetique augmente d'environ +8.6\%, alors que
                8 kB $\rightarrow$ 16 kB n'apporte qu'environ +2.7\% : le gain devient donc nettement plus marginal.
          \item A15 : \textbf{32 kB} (meilleur IPC/mW, Table~\ref{tab:q11_a15}).
                On observe encore des gains importants dans la zone haute : +8.9\% entre 8 kB $\rightarrow$ 16 kB,
                puis +16.3\% entre 16 kB $\rightarrow$ 32 kB, ce qui justifie de pousser jusqu'a 32 kB pour cette application.
        \end{itemize}

  \item \textbf{Application \texttt{blowfish}} :
        \begin{itemize}[leftmargin=*, itemsep=0.15em]
          \item A7 : \textbf{8 kB} (Table~\ref{tab:q11_a7}).
                Le saut 4 kB $\rightarrow$ 8 kB est d'environ +9.8\% d'IPC/mW, tandis que 8 kB $\rightarrow$ 16 kB
                n'apporte qu'environ +0.5\% : la courbe est presque en plateau a partir de 8 kB.
          \item A15 : \textbf{32 kB} (meilleur IPC/mW, Table~\ref{tab:q11_a15}).
                Entre 8 kB $\rightarrow$ 16 kB, le gain est faible (+2.2\%), mais 16 kB $\rightarrow$ 32 kB apporte encore
                environ +7.7\%, donc 32 kB reste le meilleur point mesure.
        \end{itemize}
\end{itemize}

\subsection{6. Facultatif}
% !TeX root = ../../rapport.tex
\subsubsection{Q13 (facultatif) --- Equivalence des choix et compromis}

\begin{formal}
\textbf{Énoncé (Q13).}
Les configurations proposees sont-elles equivalentes ? Proposer eventuellement un compromis et conclure sur les applications etudiees.
\end{formal}

\paragraph{Compromis retenu.}
Pour conserver une \emph{seule} configuration de L1 capable de supporter correctement les deux applications sur un systeme big.LITTLE, nous retenons :
\begin{itemize}[leftmargin=*, itemsep=0.15em]
  \item \textbf{A7 : L1 = 8 kB}
  \item \textbf{A15 : L1 = 32 kB}
\end{itemize}

\paragraph{Justification (energie/performance).}
Sur A7 (Table~\ref{tab:q11_a7}), augmenter de 8 kB a 16 kB apporte un gain d'IPC/mW faible : +2.7\% pour \texttt{dijkstra} et +0.5\% pour \texttt{blowfish}.
Sur A15 (Table~\ref{tab:q11_a15}), au contraire, le gain entre 16 kB et 32 kB reste important : +16.3\% pour \texttt{dijkstra} et +7.7\% pour \texttt{blowfish},
ce qui justifie le choix de 32 kB pour le coeur ``big''.

\paragraph{Justification (surface).}
Le surcout en surface totale reste modere : sur A7, passer de 8 kB a 16 kB augmente la surface d'environ +0.007 mm$^2$ ($\approx$ +0.8\%, Table~\ref{tab:q8_a7}),
et sur A15, passer de 16 kB a 32 kB augmente la surface d'environ +0.041 mm$^2$ ($\approx$ +1.7\%, Table~\ref{tab:q8_a15}).
% !TeX root = ../../rapport.tex
\subsubsection{Q14 (facultatif) --- Approche de specification pour un domaine robotique embarque}

\begin{formal}
\textbf{Énoncé (Q14).}
Proposez une approche pour la specification d'une architecture avec plusieurs applications dans un domaine specifique.
\end{formal}

\paragraph{Domaine vise.}
Nous nous placons dans le contexte \textbf{robotique embarquee} (robot mobile, drone, bras robotique) : systeme autonome avec capteurs
(camera, IMU, LiDAR, etc.), boucles de controle temps reel, et traitement local (\emph{edge}) sous contraintes d'energie et de surface.

\paragraph{Contraintes (hypotheses realistes).}
\begin{itemize}[leftmargin=*, itemsep=0.15em]
  \item \textbf{Temps reel :} certaines taches (commande moteurs, stabilisation, securite) ont des deadlines et une contrainte de jitter.
  \item \textbf{Energie/autonomie :} pour un robot mobile/donne sur batterie, minimiser l'energie par tache (ou maximiser IPC/mW) pour preserver l'autonomie.
  \item \textbf{Thermique :} souvent fanless ; privilegier des gains \emph{soutenables} (eviter une architecture qui throttle rapidement).
  \item \textbf{Surface/cout :} les caches et les coeurs ``big'' coutent en mm$^2$ (et donc en cout), il faut justifier chaque augmentation.
  \item \textbf{Memoire :} la DRAM est lente et energivore ; limiter les misses (MPKI) et la bande passante est cle (surtout pour vision/SLAM).
  \item \textbf{Surete :} degrader gracieusement (toujours garantir le controle/safety), meme si les taches lourdes (perception) saturent.
\end{itemize}

\paragraph{Ensemble representatif d'applications.}
Pour specifier une architecture, on choisit un ensemble de cas d'usage du domaine (avec des poids si besoin).
Exemple typique \textbf{robotique embarquee} :
\begin{itemize}[leftmargin=*, itemsep=0.15em]
  \item \textbf{Planification de trajectoire :} \texttt{dijkstra} / A* (graphes, acces memoire moins reguliers).
  \item \textbf{Perception :} traitement image (feature extraction) ou inference legere (CNN), charge importante et parfois vectorisable.
  \item \textbf{Localisation/SLAM :} estimation d'etat (EKF/graph-SLAM), souvent plus flottant et sensible a la memoire.
  \item \textbf{Controle :} boucles PID/commande moteurs, charge faible mais deadlines strictes.
  \item \textbf{Communication securisee :} \texttt{blowfish} (chiffrement symetrique, entiers + memoire).
\end{itemize}

\paragraph{Caracterisation des applications (ce qu'il faut mesurer).}
L'idee est d'identifier \emph{le goulot d'etranglement} de chaque application :
\begin{itemize}[leftmargin=*, itemsep=0.15em]
  \item \textbf{Mix d'instructions :} p. ex. Table~\ref{tab:iclass} montre que \texttt{dijkstra} et \texttt{blowfish} sont majoritairement
        \textbf{entiers} ($\sim$64--68\%) et \textbf{memoire} (loads+stores $\sim$32\%). En robotique, il faut aussi identifier les taches plus
        \textbf{flottantes/vectorielles} (vision, SLAM), potentiellement candidates a DSP/SIMD/NPU.
  \item \textbf{Intensite memoire :} MPKI L1/L2, taux de misses, bande passante ; distingue ``memory-bound'' vs ``compute-bound''.
  \item \textbf{Sensibilite a la taille de cache :} faire varier L1 (comme en Q4--Q5) et observer le point ou les gains deviennent marginaux.
        Dans nos resultats, \texttt{blowfish} sature vite (plateau), alors que \texttt{dijkstra} profite davantage d'une L1 plus grande.
  \item \textbf{Contraintes temps reel :} latence \emph{pire-cas} et jitter pour les boucles de controle/safety (a garantir meme sous charge).
\end{itemize}

\paragraph{Metrices et objectif d'optimisation.}
En robotique embarquee, on cherche rarement le maximum de performance brute ; on optimise plutot un compromis sous contraintes temps reel :
\begin{itemize}[leftmargin=*, itemsep=0.15em]
  \item \textbf{Contraintes ``dures'' :} deadlines/jitter pour controle et safety (verification sur le pire-cas).
  \item \textbf{Objectif ``souple'' :} maximiser un score ponderé (perf/W, perf/mm$^2$) pour perception/planification.
  \item Concretement : utiliser les metriques de Q9 (IPC/mm$^2$) et Q11 (IPC/mW), et conserver des configurations proches du front de Pareto.
\end{itemize}

\paragraph{Espace de conception (parametres a explorer).}
\begin{itemize}[leftmargin=*, itemsep=0.15em]
  \item \textbf{Heterogeneite :} big.LITTLE (un coeur ``big'' pour perception/planification + un coeur ``little'' pour controle/safety et l'energie).
  \item \textbf{Caches :} tailles L1I/L1D, associativite, taille de bloc ; L2 (souvent partagee) et sa taille.
  \item \textbf{Frequences/DVFS :} points de fonctionnement (basse/moyenne/haute) et politique de migration des taches.
\end{itemize}

\paragraph{Decision finale (exemple de methode).}
\begin{enumerate}[leftmargin=*, itemsep=0.2em]
  \item Explorer quelques tailles L1 par coeur et mesurer l'IPC (gem5), puis estimer surface/energie (CACTI + modele simple).
  \item Retenir les configurations non-dominees (Pareto) et choisir un point qui respecte les contraintes (latence/energie/surface).
  \item Specifier aussi \textbf{le mapping} : reserver le coeur ``little'' aux taches temps reel (controle/safety) pour limiter l'interference,
        et executer perception/planification sur le coeur ``big'' (et/ou accelerateurs) selon les besoins de performance.
\end{enumerate}

\paragraph{Resultat attendu.}
Au final, la specification n'est pas seulement une taille de cache : c'est une \emph{architecture + politique d'utilisation}
(type de coeurs, caches, frequences, et regles d'ordonnancement) justifiee par des mesures de performance, d'energie et de surface
sur un ensemble d'applications representatif du domaine.



\end{document}
