\subsubsection{Q2 - Comparaison de performances}

\begin{formal}
\textbf{Énoncé (Q2).}
Complétez les tableaux 9, 10 et 11. Pour cela vous devez simuler les différentes configurations et collecter les informations necessaires pour IMR, DMR et UCR.
\end{formal}

\begin{table}[htp]
    \centering
    \begin{tabular}{lcc}
        Programme & Conf. Cache - C1 & Conf. Cache - C2 \\\hline\hline
        P1(normale) & 0.000119 & 0.000119  \\
        P2(pointer) & 0.000089 & 0.000089 \\
        P3(tempo) & 0.000123 & 0.000123 \\
        P4(unrol) & 0.000141 & 0.000141 \\
    \end{tabular}
    \caption{Resultats de Instructions Miss Rate (IMR)}
    \label{tab:IMR}
    
\end{table}

Dans la Table~\ref{tab:IMR}, nous pouvons observer que les taux de miss pour les instructions (IMR) sont identiques pour les deux configurations de cache, C1 et C2. Cela suggère que les modifications apportées à la configuration de cache n'ont pas eu d'impact significatif sur le taux de miss pour les instructions. 

\begin{table}[htp]
    \centering
    \begin{tabular}{lcc}
        Programme & Conf. Cache - C1 & Conf. Cache - C2 \\\hline\hline
        P1(normale) & 0.298317 & 0.306918 \\
        P2(pointer) & 0.299730 & 0.308453 \\
        P3(tempo)   & 0.299726 & 0.308447 \\
        P4(unrol)   & 0.300091 & 0.305469 \\
    \end{tabular}
    \caption{Resultats de Data Cache Miss Rate (DMR)}
\end{table}

En ce qui concerne les taux de miss pour les données (DMR), nous pouvons observer une valeur pareille dans la configuration C2 par rapport à la configuration C1. Cela demontre que l'augmentation de l'associativité du cache de données (DL2) et du cache de niveau 2 (UL2) dans la configuration C2 n'a pas eu d'impact significatif sur le taux de miss pour les données.

\begin{table}[htp]
    \centering
    \begin{tabular}{lcc}
        Programme & Conf. Cache - C1 & Conf. Cache - C2 \\\hline\hline
        P1(normale) & 0.437203 & 0.423355 \\
        P2(pointer) & 0.437227 & 0.423262 \\
        P3(tempo)   & 0.437230 & 0.423259 \\
        P4(unrol)   & 0.434496 & 0.425145 \\
    \end{tabular}
    \caption{Resultats de Unified Cache Rate (UCR)}
\end{table}

En ce qui concerne les taux de miss pour le cache unifié (UCR), nous pouvons observer une légère amélioration dans la configuration C2 par rapport à la configuration C1. Cela suggère que l'augmentation de l'associativité du cache de données (DL2) et du cache de niveau 2 (UL2) dans la configuration C2 a eu un impact positif sur le taux de miss pour le cache unifié, bien que l'amélioration soit relativement modeste.


