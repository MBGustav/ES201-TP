\subsubsection{Q2 — Catégorie d’instructions à optimiser}

\begin{formal}
\textbf{Énoncé (Q2).}
Quelle catégorie d'instructions nécessiterait une amélioration de performances ?
Expliquez en quelques lignes (max 5 lignes).
\end{formal}

\paragraph{Réponse.}
Même si les instructions \textbf{entières} dominent (\(\sim 64\text{--}68\%\)), une part très importante provient des \textbf{accès mémoire} (loads+stores \(\approx 32\text{--}36\%\)).
La catégorie à optimiser en priorité est donc \textbf{Load/Store}, car elle dépend fortement de la \textbf{latence} et du \textbf{débit} des caches/RAM.
\textbf{BlowFish} effectue davantage de \textbf{stores} (\(\sim 12{,}7\%\) vs \(\sim 9{,}5\%\)), ce qui accentue la pression sur la hiérarchie mémoire ; améliorer cache/bande passante/préfetch est le levier le plus rentable.

% ====================================================
% 2. ÉVALUATION DES PERFORMANCES
% ====================================================
