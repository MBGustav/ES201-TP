% !TeX root = ../../rapport.tex
\subsubsection{Q1 — Pourcentage par classe d’instructions}

\begin{formal}
\textbf{Énoncé (Q1).}
Générez le pourcentage de chaque classe d'instructions de \texttt{dijkstra} et \texttt{BlowFish}
(en utilisant \texttt{gem5}) et reportez les valeurs dans un tableau.
\end{formal}

\paragraph{Objectif.}
Identifier la \emph{répartition} des classes d’instructions (contrôle, entiers, mémoire, flottants, etc.)
afin d’anticiper quelles unités fonctionnelles (ALU, LSU, unités de branchement, etc.) sont les plus sollicitées.

\paragraph{Méthodologie.}
\begin{itemize}[leftmargin=*, itemsep=0.2em]
  \item Compilation RISC-V (conforme au sujet) :
  \begin{verbatim}
REPO_ROOT=/path/to/ES201-TP
cd "$REPO_ROOT"
make -C TP4/Projet/dijkstra clean all
make -C TP4/Projet/blowfish clean all
  \end{verbatim}
  \item Simulation gem5 en mode SE (CPU OoO) --- Architecture A7 :
  \begin{verbatim}
GEM5=/path/to/gem5/build/RISCV/gem5.opt
"$GEM5" \
  -d TP4/Projet/q1_m5out/m5out_q1_a7_dijkstra TP4/se_A7.py \
  --cmd TP4/Projet/dijkstra/dijkstra_large.riscv \
  --options TP4/Projet/dijkstra/input.dat

"$GEM5" \
  -d TP4/Projet/q1_m5out/m5out_q1_a7_blowfish TP4/se_A7.py \
  --cmd TP4/Projet/blowfish/bf.riscv \
  --options e TP4/Projet/blowfish/input_large.asc \
  TP4/Projet/q1_m5out/output_q1_a7.enc 0123456789ABCDEF
  \end{verbatim}
 
  \item Extraction des pourcentages (tableau final Q1) :
  \item Extraction des compteurs par classe d’instructions depuis \texttt{m5out/stats.txt} (ou script de parsing),
        puis calcul :
        \[
          \%(\text{classe } i) = 100 \times \frac{N_i}{\sum_j N_j}.
        \]
\end{itemize}

\paragraph{Résultats.}
\begin{table}[H]
\centering
\caption{Pourcentage d’instructions par classe (gem5) — Dijkstra vs BlowFish (A7 et A15)}
\label{tab:iclass}
\scriptsize
\setlength{\tabcolsep}{3.5pt}
\renewcommand{\arraystretch}{1.10}
\begin{tabularx}{\linewidth}{>{\raggedright\arraybackslash}X *{4}{S[table-format=3.6]}}
\toprule
& \multicolumn{2}{c}{\textbf{A7}} & \multicolumn{2}{c}{\textbf{A15}} \\
\cmidrule(lr){2-3}\cmidrule(lr){4-5}
\textbf{Classe} & {\textbf{Dijkstra}} & {\textbf{BlowFish}} & {\textbf{Dijkstra}} & {\textbf{BlowFish}} \\
\midrule
Entiers (IntAlu + IntMult + IntDiv) & 68.225357 & 64.424663 & 68.225357 & 64.424663 \\
Load (MemRead)                  & 22.241504 & 22.855417 & 22.241504 & 22.855417 \\
Store (MemWrite)                &  9.533089 & 12.719215 &  9.533089 & 12.719215 \\
Contrôle (branch/jump)          &  0.000000 &  0.000000 &  0.000000 &  0.000000 \\
Flottants (Float*)              &  0.000006 &  0.000091 &  0.000006 &  0.000091 \\
Autres (No\_OpClass, etc.)      &  0.000045 &  0.000614 &  0.000045 &  0.000614 \\
\midrule
\textbf{Total}                  & 100.000001 & 100.000000 & 100.000001 & 100.000000 \\
\bottomrule
\end{tabularx}
\end{table}

\begin{itemize}[leftmargin=*, itemsep=0.2em]
  \item \textbf{Dijkstra :} Majorité d’instructions \textbf{entiers} (\(\sim 68\%\)) + \textbf{mémoire} surtout en lecture (load \(\sim 22\%\), store \(\sim 9{,}5\%\)).
  \item \textbf{BlowFish :} Entiers dominants (\(\sim 64\%\)) + \textbf{mémoire} plus marquée, avec \textbf{plus d’écritures} (store \(\sim 12{,}7\%\)) et load \(\sim 22{,}9\%\).
  \item \textbf{Point clé :} Les deux sont dominés par \textbf{entiers + accès mémoire} ; BlowFish met davantage de pression sur la mémoire (stores).
\end{itemize}
