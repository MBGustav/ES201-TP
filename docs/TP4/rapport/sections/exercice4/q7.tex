\subsubsection{Q7 — Surface L1, pourcentage et taille des coeurs}

\begin{formal}
\textbf{Énoncé (Q7).}
Quelle est la surface des caches L1 du Tableau 12 (instructions et données) en mm$^2$ ?
Quel pourcentage de la surface totale des coeurs A7 et A15 est occupé par les caches L1 ?
En déduire la taille des deux coeurs (hors caches L1). Donnez votre analyse.
\end{formal}

\paragraph{Méthode.}
Nous utilisons CACTI à 32 nm (cf. Q6), avec les paramètres L1 du Tableau 12 :
\begin{itemize}[leftmargin=*, itemsep=0.2em]
  \item A7 : 32 kB, bloc 32 B, associativité 2
  \item A15 : 32 kB, bloc 64 B, associativité 2
\end{itemize}
Les fichiers \texttt{cache\_L1\_A7\_32nm.cfg} et \texttt{cache\_L1\_A15\_32nm.cfg}
ont été créés à partir de \texttt{cache.cfg}, en modifiant uniquement les paramètres L1 utiles.
Pour Q7, on utilise seulement les champs du Tableau 12 suivants : \textbf{I-L1\$} et \textbf{D-L1\$}
(taille, taille de bloc, associativité).

Dans chaque sortie CACTI, la surface d'un cache L1 est calculée par :
\[
S_{L1} = S_{\text{data}} + S_{\text{tag}}
\]
Puis :
\[
S_{L1,\text{total}} = S_{L1I} + S_{L1D},\quad
\%L1 = \frac{S_{L1,\text{total}}}{S_{\text{coeur+L1}}}\times 100,\quad
S_{\text{coeur seul}} = S_{\text{coeur+L1}} - S_{L1,\text{total}}
\]
avec \(S_{\text{coeur+L1}}=0.45\) mm$^2$ pour A7 et \(2.0\) mm$^2$ pour A15 (énoncé).

\begin{table}[H]
\centering
\caption{Q7 — Résultats de surface (CACTI 32nm)}
\label{tab:q7_surface}
\small
\setlength{\tabcolsep}{4.5pt}
\renewcommand{\arraystretch}{1.10}
\begin{tabular}{l c c}
\toprule
\textbf{Métrique} & \textbf{A7} & \textbf{A15} \\
\midrule
\(S_{\text{data}}\) d'un L1 (mm$^2$) & 0.0301 & 0.0301 \\
\(S_{\text{tag}}\) d'un L1 (mm$^2$) & 0.0083 & 0.0044 \\
\(S_{L1I}\) (mm$^2$) & 0.0384 & 0.0346 \\
\(S_{L1D}\) (mm$^2$) & 0.0384 & 0.0346 \\
\(S_{L1,\text{total}}\) (mm$^2$) & 0.0769 & 0.0691 \\
\(\%L1\) de la surface coeur+L1 & 17.08\% & 3.46\% \\
\(S_{\text{coeur seul}}\) (mm$^2$) & 0.3731 & 1.9309 \\
\bottomrule
\end{tabular}
\end{table}

\paragraph{Analyse.}
Les L1 représentent une part nettement plus importante sur A7 (\(\approx 17.1\%\)) que sur A15 (\(\approx 3.46\%\)).
Autrement dit, la surface ``logique coeur'' (hors L1) est dominante sur A15, ce qui est cohérent avec un coeur plus large
et plus agressif micro-architecturalement. À paramètres L1 comparables (32 kB, 2-way), la différence principale
provient ici de la taille de bloc (32 B vs 64 B), qui modifie le coût de la partie tag.
