% !TeX root = ../../rapport.tex
\subsubsection{Q12 --- Choix de la configuration L1 pour un systeme big.LITTLE}

\begin{formal}
\textbf{Énoncé (Q12).}
Avec un esprit de concepteur de systeme, et en se basant sur les resultats de Q10 et Q11,
proposez la meilleure configuration du cache L1 du processeur big.LITTLE pour les applications
\texttt{dijkstra} et \texttt{blowfish} individuellement.
\end{formal}

\paragraph{Proposition (par application).}
Les gains en \% ci-dessous sont des gains relatifs d'efficacite energetique (IPC/mW) entre deux tailles L1 comparees.
\begin{itemize}[leftmargin=*, itemsep=0.25em]
  \item \textbf{Application \texttt{dijkstra}} :
        \begin{itemize}[leftmargin=*, itemsep=0.15em]
          \item A7 : \textbf{16 kB} (meilleur IPC/mW sur les points testes, Table~\ref{tab:q11_a7}).
                Entre 4 kB $\rightarrow$ 8 kB, l'efficacite energetique augmente d'environ +8.6\%, alors que
                8 kB $\rightarrow$ 16 kB n'apporte qu'environ +2.7\% : le gain devient donc nettement plus marginal.
          \item A15 : \textbf{32 kB} (meilleur IPC/mW, Table~\ref{tab:q11_a15}).
                On observe encore des gains importants dans la zone haute : +8.9\% entre 8 kB $\rightarrow$ 16 kB,
                puis +16.3\% entre 16 kB $\rightarrow$ 32 kB, ce qui justifie de pousser jusqu'a 32 kB pour cette application.
        \end{itemize}

  \item \textbf{Application \texttt{blowfish}} :
        \begin{itemize}[leftmargin=*, itemsep=0.15em]
          \item A7 : \textbf{8 kB} (Table~\ref{tab:q11_a7}).
                Le saut 4 kB $\rightarrow$ 8 kB est d'environ +9.8\% d'IPC/mW, tandis que 8 kB $\rightarrow$ 16 kB
                n'apporte qu'environ +0.5\% : la courbe est presque en plateau a partir de 8 kB.
          \item A15 : \textbf{32 kB} (meilleur IPC/mW, Table~\ref{tab:q11_a15}).
                Entre 8 kB $\rightarrow$ 16 kB, le gain est faible (+2.2\%), mais 16 kB $\rightarrow$ 32 kB apporte encore
                environ +7.7\%, donc 32 kB reste le meilleur point mesure.
        \end{itemize}
\end{itemize}