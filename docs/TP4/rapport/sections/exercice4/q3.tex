% !TeX root = ../../rapport.tex
\subsubsection{Q3 — Similitudes/divergences comportementales}

\begin{formal}
\textbf{Énoncé (Q3).}
Au regard des résultats obtenus lors du TP2, pouvez-vous justifier d’éventuelles
similitudes/divergences comportementales entre \texttt{dijkstra}, \texttt{BlowFish},
\texttt{SSCA2-BCS}, \texttt{SHA-1} et le produit de polynômes ?
\end{formal}

\paragraph{Réponse.}
L'analyse comparative des profils d'exécution entre \texttt{dijkstra}, \texttt{BlowFish}, \texttt{SSCA2-BCS},
\texttt{SHA-1} et le produit de polynômes met en évidence les tendances suivantes :

\paragraph{Convergences.}
\begin{itemize}[leftmargin=*, itemsep=0.2em]
  \item \textbf{Intensité de calcul :} Les algorithmes cryptographiques (\texttt{BlowFish} et \texttt{SHA-1})
        sollicitent fortement le processeur, principalement via de l'arithmétique entière.
  \item \textbf{Intensité des accès mémoire :} La manipulation de structures de données complexes par
        \texttt{dijkstra} et \texttt{SSCA2-BCS} se traduit par un volume important d'opérations de lecture et
        d'écriture en mémoire.
\end{itemize}

\paragraph{Divergences.}
\begin{itemize}[leftmargin=*, itemsep=0.2em]
  \item Contrairement aux flux d'exécution linéaires de \texttt{BlowFish} et \texttt{SHA-1}, l'algorithme de
        \texttt{dijkstra} se singularise par une forte densité de sauts conditionnels nécessaires à la
        détermination des chemins.
  \item Le produit de polynômes présente un profil hybride, alternant calculs et accès mémoire fréquents,
        ce qui rend critique l'optimisation des transferts de données.
\end{itemize}

\paragraph{Compléments d'analyse.}
\begin{itemize}[leftmargin=*, itemsep=0.2em]
  \item \textbf{Impact des misses :} Lorsque le CPI diminue, les défauts de cache deviennent plus pénalisants ;
        la réduction des misses (L1/L2) peut alors apporter un gain notable.
  \item \textbf{Irrégularité mémoire et préfetching :} Pour \texttt{dijkstra}/\texttt{SSCA2-BCS}, les accès
        irréguliers (\emph{pointer chasing}) rendent le préchargement moins efficace et augmentent la sensibilité
        à la hiérarchie mémoire et à l’exécution OoO.
\end{itemize}
