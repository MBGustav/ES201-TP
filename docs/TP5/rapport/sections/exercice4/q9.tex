% !TeX root = ../../rapport.tex
\subsubsection{Q9 --- Efficacite surfacique (IPC / mm$^2$) selon la taille de L1}

\begin{formal}
\textbf{Énoncé (Q9).}
En prenant en compte les deux dimensions (performance et surface) pour les deux processeurs considérés,
donnez pour chaque configuration de L1 l'efficacite surfacique de chaque processeur.
\newline\textbf{N.B. :} Efficacite surfacique = IPC / surface(mm$^2$).
\end{formal}

\paragraph{Méthode.}
Pour la performance, nous utilisons l'IPC mesuré sous gem5 :
\begin{itemize}[leftmargin=*, itemsep=0.2em]
  \item A7 : IPC issus de Q4 (tailles L1 = 1, 2, 4, 8, 16 kB)
  \item A15 : IPC issus de Q5 (tailles L1 = 2, 4, 8, 16, 32 kB)
\end{itemize}
Pour la surface, nous utilisons les surfaces totales estimées en Q8 avec CACTI :
coeur sans L1 + L1I + L1D + L2 (avec L1I = L1D dans notre balayage).
L2 est fixe a 512 kB (technologie 32 nm). Pour chaque point :
\[
\text{efficacite surfacique} = \frac{\text{IPC}}{\text{surface totale (mm$^2$)}}
\]

\begin{table}[H]
\centering
\caption{Q9 --- A7 : efficacite surfacique (IPC / mm$^2$) selon L1}
\label{tab:q9_a7}
\scriptsize
\setlength{\tabcolsep}{4.5pt}
\renewcommand{\arraystretch}{1.10}
\begin{tabular}{c c c c c c}
\toprule
\textbf{L1 (kB)} & \textbf{Surface totale (mm$^2$)} & \textbf{IPC (dijkstra)} & \textbf{IPC/mm$^2$ (dijkstra)} & \textbf{IPC (blowfish)} & \textbf{IPC/mm$^2$ (blowfish)} \\
\midrule
1  & 0.8274 & 0.2319 & 0.2802 & 0.2520 & 0.3045 \\
2  & 0.8397 & 0.2394 & 0.2851 & 0.2579 & 0.3072 \\
4  & 0.8302 & 0.2499 & 0.3010 & 0.2702 & 0.3254 \\
8  & 0.8445 & 0.2714 & 0.3214 & 0.2966 & 0.3512 \\
16 & 0.8513 & 0.2787 & 0.3274 & 0.2981 & 0.3502 \\
\bottomrule
\end{tabular}
\end{table}

\begin{table}[H]
\centering
\caption{Q9 --- A15 : efficacite surfacique (IPC / mm$^2$) selon L1}
\label{tab:q9_a15}
\scriptsize
\setlength{\tabcolsep}{4.5pt}
\renewcommand{\arraystretch}{1.10}
\begin{tabular}{c c c c c c}
\toprule
\textbf{L1 (kB)} & \textbf{Surface totale (mm$^2$)} & \textbf{IPC (dijkstra)} & \textbf{IPC/mm$^2$ (dijkstra)} & \textbf{IPC (blowfish)} & \textbf{IPC/mm$^2$ (blowfish)} \\
\midrule
2  & 2.3506 & 0.6542 & 0.2783 & 1.0601 & 0.4510 \\
4  & 2.3406 & 0.7165 & 0.3061 & 1.1343 & 0.4846 \\
8  & 2.3539 & 0.9014 & 0.3829 & 1.3597 & 0.5776 \\
16 & 2.3586 & 0.9818 & 0.4163 & 1.3901 & 0.5894 \\
32 & 2.3995 & 1.1418 & 0.4758 & 1.4975 & 0.6241 \\
\bottomrule
\end{tabular}
\end{table}

\begin{figure}[H]
\centering
\includegraphics[width=0.98\linewidth]{plot_q9_efficiency.png}
\caption{Q9 --- Efficacite surfacique (IPC / mm$^2$) en fonction de la taille L1, pour A7 (gauche) et A15 (droite).}
\label{fig:q9_eff}
\end{figure}

\paragraph{Analyse.}
La Figure~\ref{fig:q9_eff} et les Tables~\ref{tab:q9_a7} et~\ref{tab:q9_a15} montrent que l'efficacite surfacique
augmente globalement quand on augmente L1. C'est un resultat logique ici, car l'augmentation de surface totale entre deux points
reste relativement faible (la surface est dominée par le coeur sans L1 et par L2), alors que l'IPC beneficie directement
de la baisse des misses quand L1 grandit.

Pour A7, on observe clairement un effet de rendements decroissants : entre 4 kB et 8 kB, l'efficacite surfacique progresse nettement
(environ +6.8\% sur \texttt{dijkstra} et +7.9\% sur \texttt{blowfish}), alors qu'entre 8 kB et 16 kB le gain est faible
(environ +1.9\% sur \texttt{dijkstra} et quasi nul sur \texttt{blowfish}). Cela indique qu'a partir de 8 kB, une grande partie du
\emph{working set} critique tient deja dans L1 : augmenter davantage L1 apporte peu en IPC, donc peu en IPC/mm$^2$.
Ainsi, meme si 16 kB est le maximum pour \texttt{dijkstra}, 8 kB apparait comme un choix tres proche et plus \"equilibre\".

Pour A15, la tendance est plus favorable a de grandes L1 : le passage de 16 kB a 32 kB augmente l'efficacite surfacique de maniere
plus significative (environ +14.3\% sur \texttt{dijkstra} et +5.9\% sur \texttt{blowfish}), ce qui justifie que 32 kB soit la meilleure
configuration sur les points testes.
