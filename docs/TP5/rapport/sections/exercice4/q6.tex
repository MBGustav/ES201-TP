\subsubsection{Q6 — Paramètres par défaut dans \texttt{cache.cfg}}

\begin{formal}
\textbf{Énoncé (Q6).}
Observez le fichier de configuration \texttt{cache.cfg}. Quels sont les paramètres de cache
(taille, taille de bloc, associativité) et la technologie (nm) utilisés par défaut ?
\end{formal}

\paragraph{Réponse.}
D’après les lignes non commentées du fichier \texttt{cache.cfg}, la configuration par défaut est :
\begin{itemize}[leftmargin=*, itemsep=0.2em]
  \item Taille du cache : \textbf{32 kB} (\texttt{-size (bytes) 32768})
  \item Taille de bloc (ligne) : \textbf{64 B} (\texttt{-block size (bytes) 64})
  \item Associativité : \textbf{2-way} (\texttt{-associativity 2})
  \item Technologie : \textbf{32 nm} (\texttt{-technology (u) 0.032})
\end{itemize}

\paragraph{Remarque (technologie).}
Le sujet cible 28 nm, mais la version de CACTI utilisée dans ce TP ne permet pas d’obtenir un résultat stable à 28 nm
(abandon interne lors de l’exécution). Conformément à la note de l’énoncé (32 nm si 28 nm n’est pas supporté),
les estimations de surface sont donc réalisées en 32 nm (\texttt{-technology (u) 0.032}, soit 0.032 um).
