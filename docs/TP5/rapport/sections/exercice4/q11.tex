% !TeX root = ../../rapport.tex
\subsubsection{Q11 --- Efficacite energetique (IPC / mW) selon la taille de L1}

\begin{formal}
\textbf{Énoncé (Q11).}
Avec le meme protocole que precedemment, et en prenant en compte les deux dimensions (energie et surface)
pour les deux processeurs consideres, donnez pour chaque configuration de L1 l'efficacite energetique de chaque processeur
(a frequence maximale).
\newline\textbf{N.B. :} Efficacite energetique = IPC / consommation energie (mW).
\end{formal}

\paragraph{Méthode.}
Nous utilisons l'IPC mesure sous gem5 (A7: Q4, A15: Q5) et la puissance a frequence maximale calculee en Q10.
Ici, la puissance est supposee constante pour un coeur donne (d'apres les donnees du sujet), donc :
\[
\text{efficacite energetique} = \frac{\text{IPC}}{P_{\max}(\text{mW})}
\]
avec $P_{\max}=100$ mW pour A7 et $P_{\max}=500$ mW pour A15.

\begin{table}[H]
\centering
\caption{Q11 --- A7 : efficacite energetique (IPC / mW) selon L1}
\label{tab:q11_a7}
\scriptsize
\setlength{\tabcolsep}{4.5pt}
\renewcommand{\arraystretch}{1.10}
\begin{tabular}{c c c c c}
\toprule
\textbf{L1 (kB)} & \textbf{IPC (dijkstra)} & \textbf{IPC/mW (dijkstra)} & \textbf{IPC (blowfish)} & \textbf{IPC/mW (blowfish)} \\
\midrule
1  & 0.2319 & 0.002319 & 0.2520 & 0.002520 \\
2  & 0.2394 & 0.002394 & 0.2579 & 0.002579 \\
4  & 0.2499 & 0.002499 & 0.2702 & 0.002702 \\
8  & 0.2714 & 0.002714 & 0.2966 & 0.002966 \\
16 & 0.2787 & 0.002787 & 0.2981 & 0.002981 \\
\bottomrule
\end{tabular}
\end{table}

\begin{table}[H]
\centering
\caption{Q11 --- A15 : efficacite energetique (IPC / mW) selon L1}
\label{tab:q11_a15}
\scriptsize
\setlength{\tabcolsep}{4.5pt}
\renewcommand{\arraystretch}{1.10}
\begin{tabular}{c c c c c}
\toprule
\textbf{L1 (kB)} & \textbf{IPC (dijkstra)} & \textbf{IPC/mW (dijkstra)} & \textbf{IPC (blowfish)} & \textbf{IPC/mW (blowfish)} \\
\midrule
2  & 0.6542 & 0.001308 & 1.0601 & 0.002120 \\
4  & 0.7165 & 0.001433 & 1.1343 & 0.002269 \\
8  & 0.9014 & 0.001803 & 1.3597 & 0.002719 \\
16 & 0.9818 & 0.001964 & 1.3901 & 0.002780 \\
32 & 1.1418 & 0.002284 & 1.4975 & 0.002995 \\
\bottomrule
\end{tabular}
\end{table}

\begin{figure}[H]
\centering
\includegraphics[width=0.98\linewidth]{plot_q11_efficiency.png}
\caption{Q11 --- Efficacite energetique (IPC / mW) en fonction de la taille L1, pour A7 (gauche) et A15 (droite).}
\label{fig:q11_eff}
\end{figure}

\paragraph{Analyse.}
La Figure~\ref{fig:q11_eff} et les Tables~\ref{tab:q11_a7} et~\ref{tab:q11_a15} donnent directement l'efficacite energetique
(IPC/mW) pour chaque taille de L1. Dans ce modele, $P_{\max}$ est constant par coeur, donc l'evolution d'IPC/mW suit
exactement l'evolution de l'IPC.
